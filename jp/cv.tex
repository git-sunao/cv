%%%%%%%%%%%%%%%%%%%%%%%%%%%%%%%%%%%%%%%%%
% Medium Length Professional CV
% LaTeX Template
% Version 2.0 (8/5/13)
%
% This template has been downloaded from:
% http://www.LaTeXTemplates.com
%
% Original author:
% Trey Hunner (http://www.treyhunner.com/)
%
% Important note:
% This template requires the resume.cls file to be in the same directory as the
% .tex file. The resume.cls file provides the resume style used for structuring the
% document.
%
%%%%%%%%%%%%%%%%%%%%%%%%%%%%%%%%%%%%%%%%%

%----------------------------------------------------------------------------------------
%	PACKAGES AND OTHER DOCUMENT CONFIGURATIONS
%----------------------------------------------------------------------------------------

\documentclass{../sty/resume} % Use the custom resume.cls style

\usepackage[left=0.8 in,top=0.4in,right=0.8 in,bottom=0.4in]{geometry} % Document margins
\usepackage{bibentry}
\usepackage[dvipdfmx]{hyperref}
\usepackage{../sty/aas_macros}
\usepackage{etaremune}
\newcommand{\tab}[1]{\hspace{.2667\textwidth}\rlap{#1}} 
\newcommand{\itab}[1]{\hspace{0em}\rlap{#1}}
\newcommand{\myname}[1]{{\textbf{#1}}}
\newcommand{\comment}[1]{}
\newcommand{\heading}[1]{\hspace{-1.5em}\noindent\textbf{\textit{#1}}}
\hypersetup{
setpagesize=false,
bookmarksnumbered=true,
bookmarksopen=true,
colorlinks=true,
linkcolor=blue,
citecolor=red,
}

\name{杉山 素直} % Your name
\address{Compiled on \today}

\begin{document}

%----------------------------------------------------------------------------------------
%	CONTACT INFORMATION
%----------------------------------------------------------------------------------------
\begin{rSection}{連絡先}
    \begin{tabular}{ @{} >{\bfseries}l @{\hspace{6ex}} l }
    住所    & Center for Particle Cosmology, Department of Physics and Astronomy,\\
            & University of Pennsylvania, Philadelphia, PA 19104, USA \\
    部屋    & 4N21 \\
    メール  & ssunao@sas.upenn.edu \\
    ウェブサイト & \url{https://git-sunao.github.io} \\
    Github  & \url{https://github.com/git-sunao} \\
    \end{tabular}
\end{rSection}

%----------------------------------------------------------------------------------------
%	研究分野
%----------------------------------------------------------------------------------------

\begin{rSection}{研究分野}
  {\textbf{理論および観測的宇宙論}}: \\
    \textit{宇宙の大規模構造、重力レンズ(弱レンズ、マイクロレンズ)、原始ブラックホール}
\end{rSection}

%----------------------------------------------------------------------------------------
%	大型プロジェクトへの主な関与
%----------------------------------------------------------------------------------------
\begin{rSection}{共同研究}
  すばる望遠鏡HSC弱レンズグループ、メンバー (2021年~現在)\\
  ダークエネルギーサーベイ(DES)、メンバー (2024年~現在)
\end{rSection}

%----------------------------------------------------------------------------------------
%	POSITIONS
%----------------------------------------------------------------------------------------

\begin{rSection}{職歴}
  \centering
  \begin{tabular}{ @{} >{\bfseries}l @{\hspace{3ex}} p{0.8\textwidth}}
    \heading{現在}       & {\bf \href{https://www.physics.upenn.edu/people/sunao-sugiyama}{ポスドク研究員}}, \hfill 2023年9月 -- 現在 \\
                         & アメリカ合衆国, ペンシルベニア大学, フィラデルフィア \\
                         & 受入教員: Bhuvnesh Jain \\
                         & {\bf \href{https://www.jsps.go.jp/english/e-ab/index.html}{JSPS海外特別研究員}}, \hfill 2023年9月 -- 現在 \\
                         & アメリカ合衆国, ペンシルベニア大学, フィラデルフィア \\
                         & \\
    \heading{過去}       & {\bf \href{https://db.ipmu.jp/member/personal/5761en.html}{ポスドク研究員}}, \hfill 2023年4月 -- 2023年8月 \\
                         & 日本, カブリ数物連携宇宙研究機構, 千葉 \\
                         & 指導教員: 高田昌広 \\
                         & {\bf \href{https://beyondai.jp/contents/projects/murayama/?lang=en}{プロジェクト研究員}}, \hfill 2023年4月 -- 2023年8月 \\
                         & 日本, Beyond AI, 東京 \\
                         & {\bf \href{https://www.jsps.go.jp/english/e-pd/}{日本学術振興会特別研究員(DC2)}}, \hfill 2021年4月 -- 2023年3月 \\
                         & 日本, カブリ数物連携宇宙研究機構, 千葉 \\
                         & \\
    \heading{学歴}       & {\bf 東京大学, 東京, 日本}, \hfill 2020年4月 -- 2023年3月 \\
                         & 物理学専攻, 博士課程 \\
                         & 論文題目: \textit{``Joint cosmology analyses using gravitational weak lensing data from Subaru Hyper Suprime-Cam''} \\
                         & 指導教員: 高田昌広 \\
                         & {\bf 東京大学, 東京, 日本}, \hfill 2018年4月 -- 2020年3月 \\
                         & 物理学専攻, 修士 \\
                         & 論文: \textit{``Validation of cosmological analysis based on perturbation theory for wide-field galaxy survey''} \\
                         & 指導教員: 高田昌広 \\
                         & {\bf 東京大学, 東京, 日本}, \hfill 2014年4月 -- 2018年3月 \\
                         & 物理学専攻, 学士
  \end{tabular}
\end{rSection}

%----------------------------------------------------------------------------------------
%	Grant & Award
%----------------------------------------------------------------------------------------

\begin{rSection}{獲得研究資金 および 受賞}
  Grant-in-Aid for JSPS Research Fellows (DC2), Japan Society for the Promotion of Science, Apr. 2021 -- Mar. 2023

  {\textbf{\href{https://www.phys.s.u-tokyo.ac.jp/award/37776/}{理学系研究科奨励賞 (博士課程)}}}, 東京大学, 理学系, 2023年3月

  {\textbf{\href{https://www.s.u-tokyo.ac.jp/en/IGPEES/}{WINGS IGPEES, コース修了}}}, Sep. 2018 -- Mar. 2023
\end{rSection}

%----------------------------------------------------------------------------------------
% Teaching
%----------------------------------------------------------------------------------------
\begin{rSection}{教育}
  Collaborative coding: git and github, \href{https://cd3.ipmu.jp/opening/}{CD3 symposium} 2023, Kavli IPMU

  Coadvised Noriaki Nakasawa, a master student at the University of Nagoya, 2022
\end{rSection}


%----------------------------------------------------------------------------------------
%	Professional Activity
%----------------------------------------------------------------------------------------
\begin{rSection}{活動}
  \begin{tabular}{ @{} >{\bfseries}l @{\hspace{6ex}} p{0.6\textwidth}}
  学会              & 日本天文学会 (ASJ), 2018年 -- 現在 \\
                    & 日本物理学会 (JPS), 2022年 -- 現在  \\
  セミナー/ワークショップ     & IPMUランチセミナー (共同オーガナイザー), 2019年 -- 2021年 \\
                    & HSC弱レンズミニワークショップ, 2022年8月  \\
  レフェリー              & International Journal of Modern Physics D \\
  コンピューティング            & \href{https://github.com/git-sunao/fft-extended-source}{\tt fft-extended-source} \\
                    & \href{https://dark-emulator.readthedocs.io/en/latest/}{\tt dark emulator} (\href{https://darkquestcosmology.github.io}{Dark Quest Project} の一部) \\
                    & C、C++、Python、HSCパイプライン(画像解析用)を使用できます \\
  採択された観測   & \href{https://subarutelescope.org/Observing/Schedule/S20B_abstract/S20B0032abst.html}{Definitive search for PBH dark matter in the multiverse cosmology with HSC}の\textbf{PI}
  \end{tabular}
\end{rSection}

\begin{rSection}{アウトリーチ, メディア協力}
  NHK \href{https://www.nhk-ondemand.jp/goods/G2021114366SA000/}{\textit{コズミック フロント 「原始ブラックホール 宇宙創成のマスターキー」}} 出演, 2021年

  Quanta Magazine on \href{https://www.quantamagazine.org/clashing-cosmic-numbers-challenge-our-best-theory-of-the-universe-20240119/}{\textit{Clashing Cosmic Numbers Challenge Our Best Theory of the Universe}}, インタビュー, 2024年
  
  朝日新聞, \href{https://www.asahi.com/articles/ASR4H415RR48ULBH001.html}{\textit{宇宙の標準理論にほころび? 暗黒物質の精密な「地図」で解析}}, インタビュー, 2024年
\end{rSection}

%----------------------------------------------------------------------------------------
%	Publication
%----------------------------------------------------------------------------------------
\clearpage
\begin{rSection}{論文}
    最新の論文リストは\href{https://ui.adsabs.harvard.edu/search/filter_author_facet_hier_fq_author=AND&filter_author_facet_hier_fq_author=author_facet_hier%3A%221%2FSugiyama%2C%20S%2FSugiyama%2C%20Sunao%22&fq=%7B!type%3Daqp%20v%3D%24fq_author%7D&fq_author=(author_facet_hier%3A%221%2FSugiyama%2C%20S%2FSugiyama%2C%20Sunao%22)&q=pubdate%3A%5B2001-01%20TO%209999-12%5D%20author%3A(%22Sugiyama%2CSunao%22)&sort=date%20desc%2C%20bibcode%20desc&p_=0}{ADS}をご覧ください.
      \vspace{-33em}
    \nobibliography{../publists/refs}
    * = Author list alphabeticized\\
\noindent\textbf{\textit{Major author}}
\begin{enumerate}
\item \bibentry{2022PhRvD.106h3520M}
\item \bibentry{2022PhRvD.106h3519M}
\item \bibentry{2022ApJ...937...63S}
\item \bibentry{2022PhRvD.105l3537S}
\item \bibentry{2021arXiv210803063S}
\item \bibentry{2021PhLB..81436097S}
\item *\bibentry{2020PhRvL.125r1304K}
\item \bibentry{2020PhRvD.102h3520S}
\item \bibentry{2020MNRAS.493.3632S}
\item \bibentry{2019NatAs...3..524N}
\end{enumerate}

\noindent\textbf{\textit{Contributing author}}
\begin{enumerate}
\setcounter{enumi}{10}\item \bibentry{2023MNRAS.518.5171P}
\item \bibentry{2022arXiv221203257Z}
\end{enumerate}
    \bibliographystyle{../sty/etalstyle}
\end{rSection}

%----------------------------------------------------------------------------------------
%	PRESENTATIONS AT CONFERENCES, WORKSHOPS, AND MEETINGS
%----------------------------------------------------------------------------------------
\begin{rSection}{講演}
    Listing 20 selected talks among 28 talks.
\begin{enumerate}
\item \textbf{Hyper Suprime-Cam Year 3 Results: Cosmology from Weak Lensing with HSC}, \href{http://vietnam.in2p3.fr/2023/windows/index.html}{Windows on the Universe}, 2023, Aug., \textit{Oral} (\textbf{Invited Talk})
\item \textbf{HSC Year 3 Weak Lensing Cosmology Results}, \href{https://hsc-release.mtk.nao.ac.jp/doc/index.php/wly3/}{HSC webinar}, 2023, Apr., \textit{Oral}
\item \textbf{HSC Y3 cosmology results}, \href{https://www2.yukawa.kyoto-u.ac.jp/~cmb-lss/index.php}{CMB x LSS}, 2023, Apr., \textit{Oral} (\textbf{Invited Talk})
\item \textbf{Cosmology analysis with Subaru HSC Y3 data and SDSS data: cosmological parameter inference in $\Lambda$CDM model}, \href{https://www.asj.or.jp/nenkai/archive/2023a/pdf/U20a.pdf}{2023 Spring Annual Meeting of ASJ}, 2023, Mar., \textit{Oral}
\item \textbf{Cosmology analysis with Subaru HSC Y3 data and SDSS data: a joint analysis of cosmic shear + galaxy-galaxt lensing + galaxy clustering}, \href{https://www.asj.or.jp/nenkai/archive/2022b/pdf/U15a.pdf}{2022 Autumn Annual Meeting of ASJ}, 2022, Sep., \textit{Oral}
\item \textbf{Revealing the nature of dark matter with gravitational lensing: weak and microlensing}, \href{http://astro-osaka.jp/OUTAP/colloquium-abstracts.html#sugiyama}{Colloqium at Osaka theoretical astrophysics group}, 2022, Jul., \textit{Oral} (\textbf{Invited Talk})
\item \textbf{HSC cosmology: Joint analysis of galaxy-galaxy lensing and clustering from Subaru HSC and SDSS data}, \href{https://www.jps.or.jp/activities/meetings/annual/annual-index.php}{77th Annual Meeting of JPS}, 2022, Mar., \textit{Oral}
\item \textbf{Exploring Primordial black hole with microlensing observation of Andromeda galaxy}, \href{https://subarutelescope.org/Science/SubaruUM/SubaruUM2021/}{Subaru Users Meeting 2021}, 2022, Jan., \textit{Oral}
\item \textbf{Joint analysis of galaxy-galaxy lensing and clustering at large scales from Subaru HSC and SDSS data}, \href{https://sites.google.com/view/rironkon2021/}{34th astro-theory Symposium}, 2021, Dec., \textit{Oral}
\item \textbf{Joint analysis of galaxy-galaxy lensing and clustering at large scales from Subaru HSC and SDSS data}, \href{https://www.asj.or.jp/nenkai/archive/2021b/pdf/U05a.pdf}{2021 Autumn Annual Meeting of ASJ}, 2021, Sep., \textit{Oral}
\item \textbf{Exploring Dark Matter Candidates with Microlensing}, \href{https://www.kek.jp/ja/conference/20210407-3/}{KEK theory seminar}, 2021, Apr., \textit{Oral}
\item \textbf{Constraining PBH with HSC microlensing}, IPMU phenomenology lunch journal club, 2020, Dec., \textit{Oral}
\item \textbf{Testing stochastic gravitational wave signals by PBH microlensing}, \href{http://conference-indico.kek.jp/event/117/timetable/#day-2020-11-04}{4th KEK-PH + KEK-Cosmo Joint Lectures and Workshop on ``Gravitational Wave''}, 2020, Nov., \textit{Oral} (\textbf{Invited Talk})
\item \textbf{Observational constraint on PBH scenarios with HSC microlensing}, \href{https://indico.ipmu.jp/event/382/timetable/#all}{9th workshop on observational cosmology}, 2020, Nov., \textit{Oral}
\item \textbf{Developing a method of cosmological parameter inference from galaxy survey data by Subaru/HSC}, \href{http://www.astro-wakate.org/ss2019/web/}{Summer school for young researchers in astronomy/astrophysics}, 2020, Aug., \textit{Oral}
\item \textbf{Validating a minimal galaxy bias method for cosmological parameter inference using HSC-SDSS mock catalog}, Seminar at Daniel Eisenstein group@CfA, 2020, Aug., \textit{Oral}
\item \textbf{Validation of PT-based method for cosmology analysis with wide field galaxy survey data}, Seminar at astro group of Hirosaki University, 2020, Feb., \textit{Oral}
\item \textbf{Validation of PT-based method for cosmology analysis of wide field galaxy survey data}, \href{https://www.asj.or.jp/nenkai/archive/2019b/pdf/U20a.pdf}{2019 Autumn Annual Meeting of ASJ}, 2019, Sep., \textit{Oral}
\item \textbf{On the wave effect of PBH microlensing in the observation of the M31 stars}, \href{https://www.asj.or.jp/nenkai/archive/2019a/pdf/U14a.pdf}{2019 Spring Annual Meeting of ASJ}, 2019, Mar., \textit{Oral}
\item \textbf{Wave effect on PBH micro-lensing and constraintWave effect on PBH micro-lensing and constraint}, \href{http://web.cc.yamaguchi-u.ac.jp/~rsaito/obscosmo2018/}{7th workshop on observational cosmology}, 2018, Dec., \textit{Oral}
\end{enumerate}
\end{rSection}

%----------------------------------------------------------------------------------------
%	PRESS RELEASES
%----------------------------------------------------------------------------------------
\begin{rSection}{プレスリリース}
  \href{https://www.ipmu.jp/ja/20201224-PBH-multiverse}{原始ブラックホールと多元宇宙が予言するダークマターの探索}

  \href{https://www.ipmu.jp/ja/20230404-darkmatter}{ダークマターを見る! – HSC国際チームが宇宙の標準理論を検証}
\end{rSection}

\end{document}
