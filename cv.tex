%%%%%%%%%%%%%%%%%%%%%%%%%%%%%%%%%%%%%%%%%
% Medium Length Professional CV
% LaTeX Template
% Version 2.0 (8/5/13)
%
% This template has been downloaded from:
% http://www.LaTeXTemplates.com
%
% Original author:
% Trey Hunner (http://www.treyhunner.com/)
%
% Important note:
% This template requires the resume.cls file to be in the same directory as the
% .tex file. The resume.cls file provides the resume style used for structuring the
% document.
%
%%%%%%%%%%%%%%%%%%%%%%%%%%%%%%%%%%%%%%%%%

%----------------------------------------------------------------------------------------
%	PACKAGES AND OTHER DOCUMENT CONFIGURATIONS
%----------------------------------------------------------------------------------------

\documentclass{sty/resume} % Use the custom resume.cls style

\usepackage[left=0.8 in,top=0.4in,right=0.8 in,bottom=0.4in]{geometry} % Document margins
\usepackage{bibentry}
\usepackage[dvipdfmx]{hyperref}
\usepackage{sty/aas_macros}
\usepackage{etaremune}
\newcommand{\tab}[1]{\hspace{.2667\textwidth}\rlap{#1}} 
\newcommand{\itab}[1]{\hspace{0em}\rlap{#1}}
\newcommand{\myname}[1]{\textbf{#1}}
\newcommand{\comment}[1]{}
\hypersetup{
setpagesize=false,
bookmarksnumbered=true,
bookmarksopen=true,
colorlinks=true,
linkcolor=blue,
citecolor=red,
}

\name{Sunao Sugiyama} % Your name
\address{Compiled on \today}

\begin{document}

%----------------------------------------------------------------------------------------
%	CONTACT INFORMATION
%----------------------------------------------------------------------------------------
\begin{rSection}{CONTACT INFORMATION}
    \begin{tabular}{ @{} >{\bfseries}l @{\hspace{6ex}} l }
    Address & Kavli Institute for the Physics and Mathematics of the Universe (WPI), \\
            & UTIAS The University of Tokyo, Kashiwa, Chiba 277-8583, Japan \\
    Room    & A30 \\
    Email   & sunao.sugiyama@ipmu.jp \\
    Webpage & \url{https://git-sunao.github.io} \\
    Github  & \url{https://github.com/git-sunao} \\
    \end{tabular}
\end{rSection}

%----------------------------------------------------------------------------------------
%	Research Interests
%----------------------------------------------------------------------------------------

\begin{rSection}{RESEARCH INTERESTS}
    \textbf{Theoretical and Observational cosmology}\\
    large-scale structure of the Universe, gravitational weak/micro lensing, primordial black hole
\end{rSection}

%----------------------------------------------------------------------------------------
%	MAJOR INVOLVEMENT IN LARGE PROJECTS
%----------------------------------------------------------------------------------------
\begin{rSection}{MAJOR INVOLVEMENT IN LARGE PROJECTS}
    Subaru HSC weak lensing working group, member (2021-present)
\end{rSection}

%----------------------------------------------------------------------------------------
%	POSITIONS & EDUCATION SECTION
%----------------------------------------------------------------------------------------

\begin{rSection}{Positions \& Education}
{\bf Kavli IPMU, Chiba, Japan} \hfill {April 2023 -- Aug 2023}\\
Postdoctoral researcher\\
Mentor: Prof. Masahiro Takada

{\bf University of Tokyo, Tokyo, Japan}  \hfill {April 2020 -- March 2023}\\
Ph.D. course in Physics\\
Dissertation: \textit{``Joint cosmology analyses using gravitational weak lensing data from Subaru Hyper Suprime-Cam''}\\
Supervisor: Prof. Masahiro Takada

{\bf University of Tokyo, Tokyo, Japan}  \hfill {April 2018 -- March 2020}\\
M.S. in Physics\\
Dissertation: \textit{``Validation of cosmological analysis based on perturbation theory for wide-field galaxy survey''} \\
Supervisor: Prof. Masahiro Takada

{\bf University of Tokyo, Tokyo, Japan}  \hfill {April 2014 -- March 2018}\\
B.A. in Physics

\end{rSection}

%----------------------------------------------------------------------------------------
%	Award and Fellowship
%----------------------------------------------------------------------------------------

\begin{rSection}{AWARDS and FELLOWSHIP}
    \textbf{The School of Science Encouragement Award (Doctoral program)}, University of Tokyo, the School of Science, Mar. 2023
    
    \textbf{Research Fellowships for Young Scientists (Doctoral Course Students, DC2)}, Japan Society for the Promotion of Science, Apr. 2021 -- present

    \textbf{International Graduate Program for Excellence in Earth-Space Science (IGPEES)}, World-leading InnovativeGraduate Study Program (WINGS), Sep. 2018 -- present
\end{rSection}

%----------------------------------------------------------------------------------------
%	GRANTS
%----------------------------------------------------------------------------------------

\begin{rSection}{GRANTS}
    Grant-in-Aid for JSPS Research Fellows (DC2)
\end{rSection}

%----------------------------------------------------------------------------------------
%	OBSERVATIONS
%----------------------------------------------------------------------------------------
\begin{rSection}{OBSERVATIONS}
    \textbf{PI}, Definitive search for PBH dark matter in the multiverse cosmology with HSC (\href{https://subarutelescope.org/Observing/Schedule/S20B_abstract/S20B0032abst.html}{Subaru website})
\end{rSection}

%----------------------------------------------------------------------------------------
%	Professional Society
%----------------------------------------------------------------------------------------
\begin{rSection}{Professional Society}
    The Astronomical Society of Japan (ASJ), 2018 -- present

    The Physical Society of Japan (JPS), 2022 -- present
\end{rSection}

%----------------------------------------------------------------------------------------
%	Publication
%----------------------------------------------------------------------------------------
\clearpage
\begin{rSection}{PUBLICATIONS}
    For up-to-date list of my papers, please see \href{https://ui.adsabs.harvard.edu/search/filter_author_facet_hier_fq_author=AND&filter_author_facet_hier_fq_author=author_facet_hier%3A%221%2FSugiyama%2C%20S%2FSugiyama%2C%20Sunao%22&fq=%7B!type%3Daqp%20v%3D%24fq_author%7D&fq_author=(author_facet_hier%3A%221%2FSugiyama%2C%20S%2FSugiyama%2C%20Sunao%22)&q=pubdate%3A%5B2001-01%20TO%209999-12%5D%20author%3A(%22Sugiyama%2CSunao%22)&sort=date%20desc%2C%20bibcode%20desc&p_=0}{ADS}.
    \nobibliography{publists/refs}
    \vspace{-19em}
    * = Author list alphabeticized\\
\noindent\textbf{\textit{Major author}}
\begin{enumerate}
\item \bibentry{2022PhRvD.106h3520M}
\item \bibentry{2022PhRvD.106h3519M}
\item \bibentry{2022ApJ...937...63S}
\item \bibentry{2022PhRvD.105l3537S}
\item \bibentry{2021arXiv210803063S}
\item \bibentry{2021PhLB..81436097S}
\item *\bibentry{2020PhRvL.125r1304K}
\item \bibentry{2020PhRvD.102h3520S}
\item \bibentry{2020MNRAS.493.3632S}
\item \bibentry{2019NatAs...3..524N}
\end{enumerate}

\noindent\textbf{\textit{Contributing author}}
\begin{enumerate}
\setcounter{enumi}{10}
\item \bibentry{2023MNRAS.518.5171P}
\item \bibentry{2022arXiv221203257Z}
\end{enumerate}

    \bibliographystyle{sty/etalstyle}
\end{rSection}

%----------------------------------------------------------------------------------------
%	PRESENTATIONS AT CONFERENCES, WORKSHOPS, AND MEETINGS
%----------------------------------------------------------------------------------------
\begin{rSection}{Selected talks}
    Listing 20 selected talks among 28 talks.
\begin{enumerate}
\item \textbf{Hyper Suprime-Cam Year 3 Results: Cosmology from Weak Lensing with HSC}, \href{http://vietnam.in2p3.fr/2023/windows/index.html}{Windows on the Universe}, 2023, Aug., \textit{Oral} (\textbf{Invited Talk})
\item \textbf{HSC Year 3 Weak Lensing Cosmology Results}, \href{https://hsc-release.mtk.nao.ac.jp/doc/index.php/wly3/}{HSC webinar}, 2023, Apr., \textit{Oral}
\item \textbf{HSC Y3 cosmology results}, \href{https://www2.yukawa.kyoto-u.ac.jp/~cmb-lss/index.php}{CMB x LSS}, 2023, Apr., \textit{Oral} (\textbf{Invited Talk})
\item \textbf{Cosmology analysis with Subaru HSC Y3 data and SDSS data: cosmological parameter inference in $\Lambda$CDM model}, \href{https://www.asj.or.jp/nenkai/archive/2023a/pdf/U20a.pdf}{2023 Spring Annual Meeting of ASJ}, 2023, Mar., \textit{Oral}
\item \textbf{Cosmology analysis with Subaru HSC Y3 data and SDSS data: a joint analysis of cosmic shear + galaxy-galaxt lensing + galaxy clustering}, \href{https://www.asj.or.jp/nenkai/archive/2022b/pdf/U15a.pdf}{2022 Autumn Annual Meeting of ASJ}, 2022, Sep., \textit{Oral}
\item \textbf{Revealing the nature of dark matter with gravitational lensing: weak and microlensing}, \href{http://astro-osaka.jp/OUTAP/colloquium-abstracts.html#sugiyama}{Colloqium at Osaka theoretical astrophysics group}, 2022, Jul., \textit{Oral} (\textbf{Invited Talk})
\item \textbf{HSC cosmology: Joint analysis of galaxy-galaxy lensing and clustering from Subaru HSC and SDSS data}, \href{https://www.jps.or.jp/activities/meetings/annual/annual-index.php}{77th Annual Meeting of JPS}, 2022, Mar., \textit{Oral}
\item \textbf{Exploring Primordial black hole with microlensing observation of Andromeda galaxy}, \href{https://subarutelescope.org/Science/SubaruUM/SubaruUM2021/}{Subaru Users Meeting 2021}, 2022, Jan., \textit{Oral}
\item \textbf{Joint analysis of galaxy-galaxy lensing and clustering at large scales from Subaru HSC and SDSS data}, \href{https://sites.google.com/view/rironkon2021/}{34th astro-theory Symposium}, 2021, Dec., \textit{Oral}
\item \textbf{Joint analysis of galaxy-galaxy lensing and clustering at large scales from Subaru HSC and SDSS data}, \href{https://www.asj.or.jp/nenkai/archive/2021b/pdf/U05a.pdf}{2021 Autumn Annual Meeting of ASJ}, 2021, Sep., \textit{Oral}
\item \textbf{Exploring Dark Matter Candidates with Microlensing}, \href{https://www.kek.jp/ja/conference/20210407-3/}{KEK theory seminar}, 2021, Apr., \textit{Oral}
\item \textbf{Constraining PBH with HSC microlensing}, IPMU phenomenology lunch journal club, 2020, Dec., \textit{Oral}
\item \textbf{Testing stochastic gravitational wave signals by PBH microlensing}, \href{http://conference-indico.kek.jp/event/117/timetable/#day-2020-11-04}{4th KEK-PH + KEK-Cosmo Joint Lectures and Workshop on ``Gravitational Wave''}, 2020, Nov., \textit{Oral} (\textbf{Invited Talk})
\item \textbf{Observational constraint on PBH scenarios with HSC microlensing}, \href{https://indico.ipmu.jp/event/382/timetable/#all}{9th workshop on observational cosmology}, 2020, Nov., \textit{Oral}
\item \textbf{Developing a method of cosmological parameter inference from galaxy survey data by Subaru/HSC}, \href{http://www.astro-wakate.org/ss2019/web/}{Summer school for young researchers in astronomy/astrophysics}, 2020, Aug., \textit{Oral}
\item \textbf{Validating a minimal galaxy bias method for cosmological parameter inference using HSC-SDSS mock catalog}, Seminar at Daniel Eisenstein group@CfA, 2020, Aug., \textit{Oral}
\item \textbf{Validation of PT-based method for cosmology analysis with wide field galaxy survey data}, Seminar at astro group of Hirosaki University, 2020, Feb., \textit{Oral}
\item \textbf{Validation of PT-based method for cosmology analysis of wide field galaxy survey data}, \href{https://www.asj.or.jp/nenkai/archive/2019b/pdf/U20a.pdf}{2019 Autumn Annual Meeting of ASJ}, 2019, Sep., \textit{Oral}
\item \textbf{On the wave effect of PBH microlensing in the observation of the M31 stars}, \href{https://www.asj.or.jp/nenkai/archive/2019a/pdf/U14a.pdf}{2019 Spring Annual Meeting of ASJ}, 2019, Mar., \textit{Oral}
\item \textbf{Wave effect on PBH micro-lensing and constraintWave effect on PBH micro-lensing and constraint}, \href{http://web.cc.yamaguchi-u.ac.jp/~rsaito/obscosmo2018/}{7th workshop on observational cosmology}, 2018, Dec., \textit{Oral}
\end{enumerate}
\end{rSection}

%----------------------------------------------------------------------------------------
%	PEER REVIEWS
%----------------------------------------------------------------------------------------
\begin{rSection}{PEER REVIEWS}
    Reviewer of International Journal of Modern Physics D
\end{rSection}

%----------------------------------------------------------------------------------------
%	PRESS RELEASES
%----------------------------------------------------------------------------------------
\begin{rSection}{PRESS RELEASES}
    Primordial black holes and the search for dark matter from the multiverse (\href{https://www.ipmu.jp/en/20201224-PBH-multiverse}{IPMU website})
\end{rSection}

%----------------------------------------------------------------------------------------
%	PROGRAMING SKILL
%----------------------------------------------------------------------------------------
\begin{rSection}{Programming skills}
    \begin{tabular}{ @{} >{\bfseries}l @{\hspace{6ex}} l }
        Computing Language   & C, C++, Python, HSC pipeline (for image analysis)\\
        Code developed       & \href{https://github.com/git-sunao/fft-extended-source}{\tt fft-extended-source} \\
        Software Maintenance & \href{https://dark-emulator.readthedocs.io/en/latest/}{\tt dark emulator} as a part of \href{https://darkquestcosmology.github.io}{Dark Quest Project}
    \end{tabular}
\end{rSection}

%----------------------------------------------------------------------------------------
%	
%----------------------------------------------------------------------------------------
\begin{rSection}{Seminars and Workshops Organized}
    IPMU weekly lunch seminar (co-organizer), 2019 -- 2021

    HSC weaklensing mini workshop, Aug. 2022
\end{rSection}

\end{document}
