%----------------------------------------------------------------------------------------
%	CONTACT INFORMATION
%----------------------------------------------------------------------------------------
\begin{rSection}{連絡先}
    \begin{tabular}{ @{} >{\bfseries}l @{\hspace{6ex}} l }
    住所    & Center for Particle Cosmology, Department of Physics and Astronomy,\\
            & University of Pennsylvania, Philadelphia, PA 19104, USA \\
    部屋    & 4N21 \\
    メール  & ssunao@sas.upenn.edu \\
    ウェブサイト & \url{https://git-sunao.github.io} \\
    GitHub  & \url{https://github.com/git-sunao} \\
    \end{tabular}
\end{rSection}

%----------------------------------------------------------------------------------------
%	研究分野
%----------------------------------------------------------------------------------------

\begin{rSection}{研究分野}
  {\textbf{理論および観測的宇宙論}}: \\
    \textit{宇宙の大規模構造、重力レンズ(弱レンズ、マイクロレンズ)、原始ブラックホール}
\end{rSection}

%----------------------------------------------------------------------------------------
%	大型プロジェクトへの主な関与
%----------------------------------------------------------------------------------------
\begin{rSection}{共同研究}
  すばる望遠鏡HSC弱重力レンズグループ、メンバー (2021年~現在, 2024年12月より\textbf{共同議長})\\
  ダークエネルギーサーベイ(DES)、メンバー (2024年~現在)
\end{rSection}

%----------------------------------------------------------------------------------------
%	POSITIONS
%----------------------------------------------------------------------------------------

\begin{rSection}{職歴}
  \centering
  \begin{tabular}{ @{} >{\bfseries}l @{\hspace{3ex}} p{0.8\textwidth}}
    \heading{現在}       & {\bf \href{https://www.physics.upenn.edu/people/sunao-sugiyama}{ポスドク研究員}}, \hfill 2023年9月 -- 現在 \\
                         & アメリカ合衆国, ペンシルベニア大学, フィラデルフィア \\
                         & 受入教員: Bhuvnesh Jain \\
                         & {\bf \href{https://www.jsps.go.jp/english/e-ab/index.html}{JSPS海外特別研究員}}, \hfill 2023年9月 -- 現在 \\
                         & アメリカ合衆国, ペンシルベニア大学, フィラデルフィア \\
                         & \\
    \heading{過去}       & {\bf \href{https://db.ipmu.jp/member/personal/5761en.html}{ポスドク研究員}}, \hfill 2023年4月 -- 2023年8月 \\
                         & 日本, カブリ数物連携宇宙研究機構, 千葉 \\
                         & 指導教員: 高田昌広 \\
                         & {\bf \href{https://beyondai.jp/contents/projects/murayama/?lang=en}{プロジェクト研究員}}, \hfill 2023年4月 -- 2023年8月 \\
                         & 日本, Beyond AI, 東京 \\
                         & {\bf \href{https://www.jsps.go.jp/english/e-pd/}{日本学術振興会特別研究員(DC2)}}, \hfill 2021年4月 -- 2023年3月 \\
                         & 日本, カブリ数物連携宇宙研究機構, 千葉 \\
                         & \\
    \heading{学歴}       & {\bf 東京大学, 東京, 日本}, \hfill 2020年4月 -- 2023年3月 \\
                         & 物理学専攻, 博士課程 \\
                         & 論文題目: \textit{``Joint cosmology analyses using gravitational weak lensing data from Subaru Hyper Suprime-Cam''} \\
                         & 指導教員: 高田昌広 \\
                         & {\bf 東京大学, 東京, 日本}, \hfill 2018年4月 -- 2020年3月 \\
                         & 物理学専攻, 修士 \\
                         & 論文: \textit{``Validation of cosmological analysis based on perturbation theory for wide-field galaxy survey''} \\
                         & 指導教員: 高田昌広 \\
                         & {\bf 東京大学, 東京, 日本}, \hfill 2014年4月 -- 2018年3月 \\
                         & 物理学専攻, 学士
  \end{tabular}
\end{rSection}

%----------------------------------------------------------------------------------------
%	Grant & Award
%----------------------------------------------------------------------------------------

\begin{rSection}{獲得研究資金 および 受賞}
  Grant-in-Aid for JSPS Research Fellows (DC2), Japan Society for the Promotion of Science, Apr. 2021 -- Mar. 2023

  {\textbf{\href{https://www.phys.s.u-tokyo.ac.jp/award/37776/}{理学系研究科奨励賞 (博士課程)}}}, 東京大学, 理学系, 2023年3月

  {\textbf{\href{https://www.s.u-tokyo.ac.jp/en/IGPEES/}{WINGS IGPEES, コース修了}}}, Sep. 2018 -- Mar. 2023
\end{rSection}

%----------------------------------------------------------------------------------------
% Teaching
%----------------------------------------------------------------------------------------
\begin{rSection}{教育}
  Collaborative coding: git and github, \href{https://cd3.ipmu.jp/opening/}{CD3 symposium} 2023, Kavli IPMU

  Coadvised students: Rafael C. H. Gomes (a PhD student at UPenn since 2023), Noriaki Nakasawa (a master student at the University of Nagoya, 2022)
\end{rSection}


%----------------------------------------------------------------------------------------
%	Professional Activity
%----------------------------------------------------------------------------------------
\begin{rSection}{活動}
  \begin{tabular}{ @{} >{\bfseries}l @{\hspace{6ex}} p{0.6\textwidth}}
  学会                        & 日本天文学会 (ASJ), 2018年 -- 現在 \\
                              & 日本物理学会 (JPS), 2022年 -- 現在  \\
  セミナー/ワークショップ/会議     & IPMUランチセミナー (共同オーガナイザー), 2019年 -- 2021年 \\
                              & HSC弱重力レンズミニワークショップ主催, 2022年8月  \\
                              & Sesto 2025 - Tracing Cosmic Evolution with Galaxy Clusters V (SOC), 2025\\
  レフェリー                  & International Journal of Modern Physics D \\
                              & The Astrophysical Journal \\
                              & American Astronomical Society Journals \\
                              & Journal of Cosmology and Astroparticle Physics \\
  コンピューティング          & 開発コード: \href{https://github.com/git-sunao/fft-extended-source}{\tt fft-extended-source},
                                \href{https://github.com/git-sunao/fastnc}{\tt fastnc},
                                \href{https://dark-emulator.readthedocs.io/en/latest/}{\tt dark emulator} (\href{https://darkquestcosmology.github.io}{Dark Quest Project} の一部) \\
                              & C、C++、Python、HSCパイプライン(画像解析用)を使用できます \\
  採択された観測              & \href{https://subarutelescope.org/Observing/Schedule/S20B_abstract/S20B0032abst.html}{Definitive search for PBH dark matter in the multiverse cosmology with HSC}の\textbf{PI} \\
                              & \href{https://www.naoj.org/Observing/Schedule/s24b.html}{Survey of M31 eclipsing binaries: Toward a 1\% distance measurement}のco-PI
  \end{tabular}
\end{rSection}

\begin{rSection}{アウトリーチ, メディア協力}
  NHK \href{https://www.nhk-ondemand.jp/goods/G2021114366SA000/}{\textit{コズミック フロント 「原始ブラックホール 宇宙創成のマスターキー」}} 出演, 2021年

  Quanta Magazine on \href{https://www.quantamagazine.org/clashing-cosmic-numbers-challenge-our-best-theory-of-the-universe-20240119/}{\textit{Clashing Cosmic Numbers Challenge Our Best Theory of the Universe}}, インタビュー, 2024年
  
  朝日新聞, \href{https://www.asahi.com/articles/ASR4H415RR48ULBH001.html}{\textit{宇宙の標準理論にほころび? 暗黒物質の精密な「地図」で解析}}, インタビュー, 2024年
\end{rSection}
