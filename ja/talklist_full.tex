\begin{rSection}{Selected talks}
\begin{enumerate}
\item \textbf{Cosmology from Subaru HSC weak lensing Year 3 data}, \href{https://cse.umn.edu/physics/minnesota-institute-astrophysics-mifa-colloquium}{MIfA colloquium}, 2024, May., \textit{Oral} (\textbf{Invited Talk})
\item \textbf{Cosmology from weak lensing three-point correlation function}, astro/cosmo seminar at CMU, 2024, Feb., \textit{Oral}
\item \textbf{Cosmology from Subaru HSC weak lensing Year 3 data}, \href{https://www.subarutelescope.org/Science/SubaruUM/SubaruUM2023/index.html}{Subaru Users Meeting FY2023}, 2024, Jan., \textit{Oral}
\item \textbf{HSC Y3 weak lensing cosmology results}, \href{http://vietnam.in2p3.fr/2023/windows/index.html}{CosmoPalooza}, 2023, Oct., \textit{Oral}
\item \textbf{Hyper Suprime-Cam Year 3 Results: Cosmology from Weak Lensing with HSC}, \href{http://vietnam.in2p3.fr/2023/windows/index.html}{Windows on the Universe}, 2023, Aug., \textit{Oral} (\textbf{Invited Talk})
\item \textbf{HSC Year 3 Weak Lensing Cosmology Results}, \href{https://hsc-release.mtk.nao.ac.jp/doc/index.php/wly3/}{HSC webinar}, 2023, Apr., \textit{Oral}
\item \textbf{HSC Y3 cosmology results}, \href{https://www2.yukawa.kyoto-u.ac.jp/~cmb-lss/index.php}{CMB x LSS}, 2023, Apr., \textit{Oral} (\textbf{Invited Talk})
\item \textbf{Collaborative coding: git and github}, \href{https://cd3.ipmu.jp/opening/}{CD3 Opening Symposium}, 2023, Apr., \textit{Oral}
\item \textbf{すばるHSCの3年度データとSDSSデータを用いた宇宙論解析: ΛCDMモデルにおける宇宙論パラメタ推定}, \href{https://www.asj.or.jp/nenkai/archive/2023a/pdf/U20a.pdf}{天文学会2023年春季年会}, 2023, Mar., \textit{Oral}
\item \textbf{すばるHSCの弱い重力レンズによる宇宙論}, \href{http://gopira.jp/Dthesis2022/program.html}{2023年光赤天連学位論文発表会}, 2023, Mar., \textit{Oral}
\item \textbf{すばるHSCの3年度データとSDSSデータを用いた宇宙論解析:弱重力レンズ+銀河-弱重力レンズ+銀河クラスタリングの統合解析}, \href{https://www.asj.or.jp/nenkai/archive/2022b/pdf/U15a.pdf}{天文学会2022年秋季年会}, 2022, Sep., \textit{Oral}
\item \textbf{Revealing the nature of dark matter with gravitational lensing: weak and microlensing}, \href{http://astro-osaka.jp/OUTAP/colloquium-abstracts.html#sugiyama}{Colloqium at Osaka theoretical astrophysics group}, 2022, Jul., \textit{Oral} (\textbf{Invited Talk})
\item \textbf{HSC宇宙論:すばるHSCとSDSSデータの銀河弱重力レンズとクラスタリングの信号を用いた宇宙論統合解析 }, \href{https://www.jps.or.jp/activities/meetings/annual/annual-index.php}{日本物理学会第77回年次大会}, 2022, Mar., \textit{Oral}
\item \textbf{Exploring Primordial black hole with microlensing observation of Andromeda galaxy}, \href{https://subarutelescope.org/Science/SubaruUM/SubaruUM2021/}{Subaru Users Meeting 2021}, 2022, Jan., \textit{Oral}
\item \textbf{すばるHSCとSDSSデータの銀河弱重力レンズとクラスタリングの大スケール信号を用いた宇宙論統合解析}, \href{https://sites.google.com/view/rironkon2021/}{第34回理論懇シンポジウム}, 2021, Dec., \textit{Oral}
\item \textbf{すばるHSCとSDSSデータの銀河弱重力レンズとクラスタリングの大スケール信号を用いた宇宙論統合解析}, \href{https://sites.google.com/view/obscosmws2021main}{第10回観測的宇宙論ワークショップ}, 2021, Nov., \textit{Oral}
\item \textbf{すばるHSCとSDSSデータの銀河弱重力レンズとクラスタリングの大スケール信号を用いた宇宙論統合解析}, \href{https://www.asj.or.jp/nenkai/archive/2021b/pdf/U05a.pdf}{天文学会2021年秋季年会}, 2021, Sep., \textit{Oral}
\item \textbf{Exploring Dark Matter Candidates with Microlensing}, \href{https://www.kek.jp/ja/conference/20210407-3/}{KEK theory seminar}, 2021, Apr., \textit{Oral}
\item \textbf{Constraining PBH with HSC microlensing}, IPMU phenomenology lunch journal club, 2020, Dec., \textit{Oral}
\item \textbf{Testing stochastic gravitational wave signals by PBH microlensing}, \href{http://conference-indico.kek.jp/event/117/timetable/#day-2020-11-04}{4th KEK-PH + KEK-Cosmo Joint Lectures and Workshop on ``Gravitational Wave''}, 2020, Nov., \textit{Oral} (\textbf{Invited Talk})
\item \textbf{HSCマイクロレンズによるPBHシナリオの観測的制限}, \href{https://indico.ipmu.jp/event/382/timetable/#all}{第9回観測的宇宙論ワークショップ}, 2020, Nov., \textit{Oral}
\item \textbf{すばる HSC の銀河サーベイデータを使っ た宇宙論パラメタ推定手法の開発}, \href{http://www.astro-wakate.org/ss2019/web/}{2019天文・天体物理若手夏の学校}, 2020, Aug., \textit{Oral}
\item \textbf{Validating a minimal galaxy bias method for cosmological parameter inference using HSC-SDSS mock catalog}, Seminar at Daniel Eisenstein group@CfA, 2020, Aug., \textit{Oral}
\item \textbf{摂動論的手法の検証と HSC 初年度データからの宇宙論パラメタの制限}, \href{http://www.asj.or.jp/nenkai/archive/2020a/pdf/U03a.pdf}{天文学会2020年春季年会}, 2020, Mar.
\item \textbf{Constraints on Primordial Black Holes with Microlensing}, Informal seminar at Takahashi and Asada Labs, 2020, Feb., \textit{Oral}
\item \textbf{広天域銀河サーベイデータの宇宙論解析における摂動論的手法の有効性の検証}, Seminar at astro group of Hirosaki University, 2020, Feb., \textit{Oral}
\item \textbf{Constraints on Primordial Black Holes with Microlensing: Wave \& Finite Source Effects / PBH from Multiverse}, \href{http://indico.ipmu.jp/event/313/overview}{Berkeley Week at Kavli IPMU}, 2020, Jan., \textit{Oral}
\item \textbf{広天域銀河サーベイデータの宇宙論解析 における摂動論的手法の有効性の検証}, \href{https://www.asj.or.jp/nenkai/archive/2019b/pdf/U20a.pdf}{天文学会2019年秋季年会}, 2019, Sep., \textit{Oral}
\item \textbf{Test and validation of PT-based cosmology : g-g lensing and clustering}, \href{http://pt-chat-kyoto.sciencesconf.org/}{PT chat}, 2019, Apr., \textit{Poster}
\item \textbf{M31 星に対する原始ブラックホールのマイクロレンジングへの波動効果の影響}, \href{https://www.asj.or.jp/nenkai/archive/2019a/pdf/U14a.pdf}{天文学会2019年春季年会}, 2019, Mar., \textit{Oral}
\item \textbf{Wave Effect on PBH Microlensing}, \href{http://www2.yukawa.kyoto-u.ac.jp/~aud2019/index.php}{Accelerating universe in the dark}, 2019, Mar., \textit{Poster}
\item \textbf{Wave effect on PBH micro-lensing and constraintWave effect on PBH micro-lensing and constraint}, \href{http://web.cc.yamaguchi-u.ac.jp/~rsaito/obscosmo2018/}{第7回観測的宇宙論ワークショップ}, 2018, Dec., \textit{Oral}
\item \textbf{BAO 復元アルゴリズムの提案と評価BAO 復元アルゴリズムの提案と評価}, \href{http://www.astro-wakate.org/ss2018/web/link.html}{2018天文・天体物理若手夏の学校}, 2018, Aug., \textit{Oral}
\end{enumerate}\end{rSection}
