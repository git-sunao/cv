\begin{rSection}{講演}
全42件のうち19件のトークを選出しました。
全リストは\href{https://github.com/git-sunao/cv/blob/main/ja/sunao_type_list.pdf}{こちら}をご覧ください。\begin{enumerate}
\item \textbf{Exploring Primordial Black Hole with Microlensing Data: Updates on Analysis Pipeline}, UPenn CfPC workshop, 2024, Nov., \textit{Oral}
\item \textbf{Exploring Primordial Black Hole with Microlensing Data: Updates on Analysis Pipeline}, \href{https://indico.ipmu.jp/event/439/overview}{Focus week on primordial black holes 2024}, 2024, Nov., \textit{Oral} (\textbf{Invited Talk})
\item \textbf{Cosmology with third-order shear statistics}, Roman F2F meeting, 2024, Oct., \textit{Oral}
\item \textbf{Exploring Primordial Black Hole with Microlensing Data}, \href{https://pacific-conference.pa.ucla.edu/index.html}{Pacific conference}, 2024, Aug., \textit{Oral} (\textbf{Invited Talk})
\item \textbf{Cosmology from Subaru HSC weak lensing Year 3 data}, \href{https://cse.umn.edu/physics/minnesota-institute-astrophysics-mifa-colloquium}{MIfA colloquium}, 2024, May., \textit{Oral} (\textbf{Invited Talk})
\item \textbf{Cosmology from weak lensing three-point correlation function}, astro/cosmo seminar at CMU, 2024, Feb., \textit{Oral}
\item \textbf{Cosmology from Subaru HSC weak lensing Year 3 data}, \href{https://www.subarutelescope.org/Science/SubaruUM/SubaruUM2023/index.html}{Subaru Users Meeting FY2023}, 2024, Jan., \textit{Oral}
\item \textbf{HSC Y3 weak lensing cosmology results}, \href{http://vietnam.in2p3.fr/2023/windows/index.html}{CosmoPalooza}, 2023, Oct., \textit{Oral}
\item \textbf{HSC Year 3 Weak Lensing Cosmology Results}, \href{https://hsc-release.mtk.nao.ac.jp/doc/index.php/wly3/}{HSC webinar}, 2023, Apr., \textit{Oral}
\item \textbf{HSC Y3 cosmology results}, \href{https://www2.yukawa.kyoto-u.ac.jp/~cmb-lss/index.php}{CMB x LSS}, 2023, Apr., \textit{Oral} (\textbf{Invited Talk})
\item \textbf{Collaborative coding: git and github}, \href{https://cd3.ipmu.jp/opening/}{CD3 Opening Symposium}, 2023, Apr., \textit{Oral}
\item \textbf{Revealing the nature of dark matter with gravitational lensing: weak and microlensing}, \href{http://astro-osaka.jp/OUTAP/colloquium-abstracts.html#sugiyama}{Colloqium at Osaka theoretical astrophysics group}, 2022, Jul., \textit{Oral} (\textbf{Invited Talk})
\item \textbf{Exploring Primordial black hole with microlensing observation of Andromeda galaxy}, \href{https://subarutelescope.org/Science/SubaruUM/SubaruUM2021/}{Subaru Users Meeting 2021}, 2022, Jan., \textit{Oral}
\item \textbf{すばるHSCとSDSSデータの銀河弱重力レンズとクラスタリングの大スケール信号を用いた宇宙論統合解析}, \href{https://www.asj.or.jp/nenkai/archive/2021b/pdf/U05a.pdf}{天文学会2021年秋季年会}, 2021, Sep., \textit{Oral}
\item \textbf{Exploring Dark Matter Candidates with Microlensing}, \href{https://www.kek.jp/ja/conference/20210407-3/}{KEK theory seminar}, 2021, Apr., \textit{Oral}
\item \textbf{Testing stochastic gravitational wave signals by PBH microlensing}, \href{http://conference-indico.kek.jp/event/117/timetable/#day-2020-11-04}{4th KEK-PH + KEK-Cosmo Joint Lectures and Workshop on ``Gravitational Wave''}, 2020, Nov., \textit{Oral} (\textbf{Invited Talk})
\item \textbf{HSCマイクロレンズによるPBHシナリオの観測的制限}, \href{https://indico.ipmu.jp/event/382/timetable/#all}{第9回観測的宇宙論ワークショップ}, 2020, Nov., \textit{Oral}
\item \textbf{広天域銀河サーベイデータの宇宙論解析における摂動論的手法の有効性の検証}, Seminar at astro group of Hirosaki University, 2020, Feb., \textit{Oral}
\item \textbf{Wave effect on PBH micro-lensing and constraintWave effect on PBH micro-lensing and constraint}, \href{http://web.cc.yamaguchi-u.ac.jp/~rsaito/obscosmo2018/}{第7回観測的宇宙論ワークショップ}, 2018, Dec., \textit{Oral}
\end{enumerate}\end{rSection}
